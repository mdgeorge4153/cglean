\newtheorem{definition}{Definition}
\newtheorem{theorem}{Theorem}
\newtheorem{lemma}{Lemma}
\newtheorem{corollary}{Corollary}

\chapter{Overview}

\chapter{2D Planar Geometry Primitives}

\chapter{Planar Arrangements}

\chapter{Point Location Queries}

\chapter{Minkowski Sum}

\chapter{Computational Number Representations}

For our computational geometry algorithms to be correct, we need to use exact
number representations --- roundoff error introduced by floating-point
approximations can easily lead to incorrect outcomes.  For example,
if the intersection of two polygons is a single point or line, small rounding
errors could easily make the intersection look empty.

Mathlib provides an exact real number representation that encodes a number as
an infinite sequence of rational approximations. Unfortunately, this
representation is not suitable for computation because some operations are not
computable. For example, determining whether two reals are equal requires
iterating over the entire infinite sequences that represent them.

The rational numbers ($ℚ$) are a reasonable choice of number type: fractions are easy
to compute with and don't introduce any rounding error. Some applications
require irrational numbers though. For example, rotating a unit line segment by
$45°$ produces a point with coordinates $(\sqrt{2}/2, \sqrt{2}/2)$, whose
coordinates are irrational.

If we only cared about $45°$ rotations, we could store numbers of the
form $a_1 + a_2\sqrt{2}$ where $a_1$ and $a_2$ are rational numbers. It
turns out that adding, multiplying, and dividing numbers of this form produces
numbers of this form, and it is not too hard to compare them for equality and
inequalities; these operations are sufficient to perform many geometric
operations (including polygon union and intersection).

Of course if you give a mouse a $45°$ rotation, they will want a
rotation of $30°$. We can support both $45°$ and $30°$
rotations with numbers of the form
$a_1 + a_2\sqrt{2} + a_3\sqrt{3} + a_6\sqrt{6}$; these numbers also support the
important operations for many geometric algorithms.

This suggests that the choice of number representation should be application
specific, and the geometric algorithms should be compatible with different
number representations. This package constructs a few concrete number
representations that applications can use, and proves that they are correct.

In this context, correctness means that the numbers support the necessary
operations, and that those operations behave as you would expect. More
concretely, we show that the number representations form ``linearly ordered
fields,'' which are types that support addition, multiplication, subtraction,
division, and computable comparisons (equality and inequality), all of which
must satisfy certain axioms.

The first implementation is \code{AdjoinRoot A n} (abbreviated $A[\sqrt{n}]$),
which represents numbers of the form $a_1 + a_n\sqrt{n}$. Instead of requiring
$a_1$ and $a_n$ to be rationals, we allow them to be taken from any fixed
linearly ordered field $A$.  This allows the construction to be repeated. For
example, we can factor numbers of the form $a_1 + a_2\sqrt{2} + a_3\sqrt{3} +
a_6\sqrt{6}$ as $(a_1 + a_2\sqrt{2}) + (a_3 + a_6\sqrt{2})\sqrt{3}$, and then
represent them using $ℚ[\sqrt{2}][\sqrt{3}]$.

The second implementation is intended to improve performance by using
floating-point approximations wherever possible, and only falling back on the
exact computations when the floating-point arithmetic is not precise enough. We
define \code{FilteredEmbedding α} which wraps deferred values of $α$
inside a floating-point interval.

We can summarize the example from this section in the following corollary:
\begin{corollary}
  \label{thm:numberType}
  \uses{def:adjoin, def:filtered}
  \code{FilteredEmbedding $ℚ[\sqrt{2}][\sqrt{3}]$} is a linearly ordered field.
\end{corollary}

\begin{proof}
  \uses{thm:adjoinOrderedField, thm:filteredOrderedField, thm:adjoinToRealOrderedFieldHom}
  TODO
\end{proof}

The definition of a linearly ordered field can be broken down into parts, and
these parts are important stepping stones on the path to proving that a number
representation is a linearly ordered field. In particular, we will mention the
following intermediate interfaces:

\begin{enumerate}
\item A (commutative) ring must support addition, subtraction, and
multiplication, but not division. For example, the integers ($ℤ$) are a ring
but not a field. Rings are useful in this context because one can generate a
field from a ring by forming fractions; an alternative approach to defining
$ℚ[\sqrt{2}][\sqrt{3}]$ would be to take the field of fractions of
$ℤ[\sqrt{2}][\sqrt{3}]$.

\item A field is a ring that also supports division.

\item A linear order is a type that supports comparisons ($≤$, $<$, etc).

\item A linearly ordered ring is a ring with a linear order that is compatible
with the ring operations; a linearly ordered field is a field with a linear
order that is compatible with the field operations.
\end{enumerate}

Note that we are borrowing all of these definitions from Mathlib, so we are
leaving the exact definitions a bit vague here.  These definitions can also be
broken down further (e.g. Monoid, Group, Partial Order, etc) and we do prove
related intermediate results where possible, but these are probably less useful
for this library so we don't describe them here.

\section{Adjoining a square root}

%% Definition

\begin{definition}
  \label{def:adjoin}
  \lean{AdjoinSqrt}
  \leanok
  $A[\sqrt{n}]$ is stored as a pair $(a_1, a_n)$, representing the value
  $a_1 + a_n\sqrt{n}$.
\end{definition}

%% A[√n] is a ring

\begin{lemma}
  \label{thm:adjoinRing}
  \lean{instAdjoinCommRing}
  \leanok
  \uses{def:adjoin}
  If $A$ is a (commutative) ring, then we can provide a commutative ring structure
  on $A[\sqrt{n}]$ in the obvious way.
\end{lemma}

\begin{proof}
  TODO
\end{proof}

%% A[√n] has √n

Of course, the whole point of $A[√n]$ is that it contains a square root of $n$:
\begin{theorem}
  \label{thm:adjoinHasSqrtN}
  \lean{AdjoinSqrt.rootN}
  \uses{def:adjoin, thm:adjoinRing}
  If $A$ has multiplication, then in $A[\sqrt{n}]$, $\sqrt{n} * \sqrt{n} = n$
\end{theorem}

%% A[√n] is a field

\begin{lemma}
  \label{thm:adjoinField}
  \lean{instAdjoinField}
  \leanok
  \uses{def:adjoin, thm:adjoinRing}
  If $A$ is a field and $n$ is not a square in $A$, then $A[\sqrt{n}]$ is a
  field.
\end{lemma}

%% A[√n] is ordered ring

\begin{lemma}
  \label{thm:adjoinOrderedRing}
  \lean{instAdjoinLinearOrderedRing}
  \uses{def:adjoin, thm:adjoinRing}

  If $A$ is a linearly ordered ring, and if $n$ is a positive non-square in $A$
  then $A[\sqrt{n}]$ is also linearly ordered.
\end{lemma}

\begin{proof}
  TODO
\end{proof}

%% A[√n] is an ordered field

\begin{theorem}
  \label{thm:adjoinOrderedField}
  \lean{instAdjoinLinearOrderedField}
  \uses{def:adjoin, thm:adjoinField, thm:adjoinOrderedRing}

  If $A$ is a linearly ordered field, and if $n$ is a positive non-square in $A$
  then $A[\sqrt{n}]$ is also a linearly ordered field.
\end{theorem}
\begin{proof}
  TODO
\end{proof}

%%%%%%%%%%%%%%%%%%%%%%%%%%%%%%%%%%%%%%%%%%%%%%%%%%%%%%%%%%%%%%%%%%%%%%%%%%%%%%%%
\subsection{Converting from $A[\sqrt{n}]$ to $ℝ$}

In order to use $A[\sqrt{n}]$ with \code{FilteredEmbedding}, we need to have a way of
converting values in $A[\sqrt{n}]$ to real numbers, and this conversion has to preserve
the operations defined on $A[\sqrt{n}]$.

TODO: informal definitions for homs.

%% A[√n] → ℝ

\begin{definition}
  \label{def:adjoinToReal}
  \lean{AdjoinRoot.toReal}
  \uses{def:adjoin}
  Given a function $toReal : A → ℝ$ and $n ∈ A$ with $n.toReal \geq 0$,
  there is a corresponding function $AdjoinRoot.toReal : A[√n] → ℝ$ given by
  $(a_1 + a_n\sqrt{n}).toReal := a_1.toReal + a_n.toReal * \sqrt{n.toReal}$.
\end{definition}

\begin{lemma}
  \label{thm:adjoinToRealRingHom}
  \uses{def:adjoinToReal, thm:adjoinRing}
  If $A$ is a ring and $toReal : A → ℝ$ is a ring homomorphism, then $AdjoinRoot.toReal$ is a ring
  homomorphism.
\end{lemma}

\begin{lemma}
  \label{thm:adjoinToRealOrderHom}
  \uses{def:adjoinToReal, thm:adjoinOrderedRing}
  If $A$ is an ordered ring and $toReal : A → ℝ$ is monotonic then $AdjoinRoot.toReal$ is also monotonic.
\end{lemma}


\begin{lemma}
  \label{thm:adjoinToRealOrderedFieldHom}
  \uses{def:adjoinToReal, thm:adjoinOrderedField, thm:adjoinToRealRingHom, thm:adjoinToRealOrderHom}
  If $A$ is an ordered field and $toReal : A → ℝ$ is an ordered field homomorphism
  then $AdjoinRoot.toReal$ is also an ordered field homomorphism.
\end{lemma}

%%%%%%%%%%%%%%%%%%%%%%%%%%%%%%%%%%%%%%%%%%%%%%%%%%%%%%%%%%%%%%%%%%%%%%%%%%%%%%%%
\subsection{Converting from $A$ to $A[√n]$}

These are some interesting results that aren't directly needed to use these
number representations, but are interesting nonetheless.

%% A → A[√n]

\begin{definition}
  \label{def:adjoinEmbedding}
  \lean{instCoeAdjoin}
  \leanok
  \uses{def:adjoin}
  We write $A \hookrightarrow A[\sqrt{n}]$ for the natural embedding
  $a \mapsto a + 0\sqrt{n}$ (assuming $0 ∈ A$).
\end{definition}

%% A[√n] is an A-algebra

\begin{lemma}
  \label{thm:adjoinAlgebra}
  \lean{instAdjoinAlgebra}
  \leanok
  \uses{def:adjoin, def:adjoinEmbedding, thm:adjoinRing}
  If $A$ is a ring, then $A[\sqrt{n}]$ is an $A$-algebra.
\end{lemma}

\begin{lemma}
  \label{thm:mathlibAdjoin}
  \uses{def:adjoin}
  $A[\sqrt{n}]$ is equivalent to $A[x] / (x^2 - n)$ as defined in mathlib.
\end{lemma}

%% A → A[√n] is injective

\begin{lemma}
  \label{thm:adjoinInjective}
  \lean{instAdjoinInjective}
  \uses{def:adjoinEmbedding}
  If $n$ is not a square in $A$ then the inclusion $A \hookrightarrow
  A[\sqrt{n}]$ is injective.
\end{lemma}

%%%%%%%%%%%%%%%%%%%%%%%%%%%%%%%%%%%%%%%%%%%%%%%%%%%%%%%%%%%%%%%%%%%%%%%%%%%%%%%%
\section{Interval arithmetic}

\begin{definition}
  \label{def:filtered}
  \leanok
  \lean{FilteredEmbedding}
  Given a function $f : α → ℝ$, a \code{FilteredEmbedding f} value consists of
  an unevaluated value of type $α$ and a closed floating-point interval that
  contains $f α$. We consider two values in \code{FilteredEmbedding f} to be
  equal if the wrapped $α$ values are equal; the intervals are just an
  approximation.
\end{definition}

\begin{lemma}
  \label{thm:filteredRing}
  \lean{instFilteredEmbeddingRing}
  \uses{def:filtered}
  If $A$ is a ring and $f : A → ℝ$ is a ring homomorphism then \code{FilteredEmbedding f}
  is a ring.
\end{lemma}

\begin{lemma}
  \label{thm:filteredField}
  \uses{thm:filteredRing}
  \lean{instFilteredEmbeddingField}
  If $A$ is a field and $f : A → ℝ$ is a field homomorphism then \code{FilteredEmbedding f}
  is a field.
\end{lemma}

\begin{lemma}
  \label{thm:filteredOrderedRing}
  \uses{thm:filteredRing}
  \lean{instFilteredEmbeddingLinearOrderedRing}
  If $A$ is a linearly ordered ring and if $f : A → ℝ$ is an ordered ring homomorphism then
  \code{FilteredEmbedding f} is a linearly ordered ring.
\end{lemma}

\begin{theorem}
  \label{thm:filteredOrderedField}
  \lean{instFilteredEmbeddingLinearOrderedField}
  \uses{def:filtered, thm:filteredOrderedRing, thm:filteredField}
  If $α$ is a linearly ordered field and $f : α → ℝ$ preserves all of the field
  and ordering properties of $α$, then \code{FilteredEmbedding f} is a linearly
  ordered field.
\end{theorem}

\subsection{Converting between $A$ and \code{FilteredEmbedding (f : A → ℝ)}}

The whole point of \code{FilteredEmbedding} is that it encapsulates values of
$A$.

\begin{definition}
  \label{def:filteredEquivalent}
  \uses{def:filtered}
  \notready
  There is a nice injection $i$ from \code{FilteredEmbedding (f : A → ℝ)} to
  $A$, and $f \circ i = toReal$.
\end{definition}

\subsection{FilteredReal avoids computation}

There's no way to prove in lean that the underlying $A$ values are not computed
unless strictly necessary, but we can at least prove that the interval
comparison only fails if the comparison can't be determined just from the
intervals. We formalize this as follows:

\begin{lemma}
  \label{thm:compareComplete}
  \uses{def:filtered}
  \lean{Interval.compare?_complete}
  If the function we use to compare intervals indicates that they are
  incomparable, then we can choose two sets of points in the two intervals that
  would give different comparison results.
\end{lemma}

\chapter{Applications}

\section{Collision-Avoiding Drag and Drop}

\section{Origami}



% vim: ts=2 sw=2 et ai
